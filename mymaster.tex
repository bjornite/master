\documentclass[UKenglish]{ifimaster}  %% ... or USenglish or norsk or nynorsk
\usepackage[utf8]{inputenc}           %% ... or latin1 or applemac
\usepackage[T1]{fontenc,url}
\urlstyle{sf}
\usepackage{babel,textcomp,csquotes,duomasterforside,varioref,graphicx}
\usepackage[backend=biber,style=numeric-comp]{biblatex}
\usepackage[pdftex,pdfpagelabels,bookmarks,hyperindex,hyperfigures]{hyperref}

\title{The title of my thesis}        %% ... or whatever
\subtitle{Any short subtitle}         %% ... if any
\author{Bjørn Ivar Teigen}                      %% ... or whoever 

\bibliography{mybib}                  %% ... or whatever

\begin{document}
\duoforside[dept={Department of Informatics},   %% ... or your department
  program={Robotics and intelligent systems},  %% ... or your programme
  long]                                        %% ... or short

\frontmatter{}

\chapter*{Abstract}                   %% ... or Sammendrag or Samandrag

\tableofcontents{}
\listoffigures{}
\listoftables{}

\chapter*{Preface}                    %% ... or Forord

\mainmatter{}
\part{Introduction}                   %% ... or Innledning or Innleiing

\chapter{Background}                  %% ... or Bakgrunn
Writing is about structuring your thoughts, not about being poetic.
Writing is good for your work

\begin{tabular}{l}
  enables you to collaborate with others \\
  Writing has parallells to research
\end{tabular}

Convolutional neural networks

Explained to a family member:

Convolutional neural networks consist of artificial neurons that are connected to each other.
The network consists of several layers placed on top of each other. The layers are connected in such a way that each neuron is
connected to a square of neurons in the layer below it. This mimics the way real neurons are connected to the sensory cells in the retina of the eye.

Explained to a fellow student:

The network consists of several layers placed on top of each other. The layers are connected in such a way that each neuron is
connected to a square of neurons in the layer below it. The weigths from each square are the same, so that the number of parameters
for each layer is proportional to the size of the square and the number of neurons in the receiving layer. This reduces the number of
parameters compared to a fully connected network.

Explained to my professor:

\chapter{Writing course}
How to edit:

\begin{tabular}{l}
1. Why am I reading this? \\
2. Where do I fall off?\\
3. Does everything make sense? Is it logical?\\
4. What questions pop up when reading?\\
5. Typos and words
\end{tabular}


\section{Is AI dangerous?}

We're witnessing a revolution in the field of AI. Some argue that the AI's will one day surpass our
human intellect, become conscious, and dominate the world. Others say that is impossible.

A computer program is just a set of 0s and 1s, interpreted by a computer in a deterministic way. It's
fundamentally no different from any other machine in the ways they function. What seperates a computer from other machines are the fact that they process information in a completely general sense. Any formally defineable process, that is any process we can create a mathematical model of, can be simluated by a computer. It follows, that in order for the human mind to be un-simulateable, it has to be impossible to formally model the brain.

The brain is a complex web of neurons and other cells, and on the cell level we have a pretty good understanding of how it works. We can simulate the processes in a computer. There is no indication that any kind of brain cell breaks the known laws of physics, chemistry or biology. Thus, in order for the mind to be un-simulateable, there has to be some way in which the brain cells are connected, or in the way they communicate, which is impossible to model. This seems completely implausible to me, and in case I'm correct, it should be possible to simulate a human mind. That does not mean it's a simple task!




\begin{tabular}{l}
  
\end{tabular}

\section{Machine learning}
\subsection{Reinforcement learning}
\subsubsection{Autonomous agents}
\paragraph{Reward functions}
\subparagraph{some math}

\section{Reinforcement learning}
RL is the process of optimizing the behaviour of a system based on experience obtained through trial and error. \cite{boka jeg fikk av kai}
\section{Motivation}

\part{The project}                    %% ... or ??

\chapter{Planning the project}        %% ... or ??


\part{Conclusion}                     %% ... or Konklusjon

\chapter{Results}                     %% ... or ??


\backmatter{}
\printbibliography
\end{document}
